\documentclass[10pt]{curriculum}

\columnratio{0.55, 0.45} % Widths of the two columns, specified here as a ratio summing to 1 to correspond to percentages; adjust as needed for your content 

% Headers and footers can be added with the following commands: \lhead{}, \rhead{}, \lfoot{} and \rfoot{}
% Example right footer:
%\rfoot{\textcolor{headings}{\sffamily Last update: \today. Typeset with Xe\LaTeX}}

%----------------------------------------------------------------------------------------

\begin{document}

\begin{center}
	{\sffamily\Huge Luis Miguel Casañ González}
	
	\medskip
	
	{\Huge{Curriculum Vitae}}
\end{center}

BSc. Computer Science with a lot of desire to learn. Open‑source and Linux enthusiast. I have tried a
lot of different software and technologies and I have worked as a backend developer in some small projects. Math lover and beginning to learn AI. Creative and motivated to propose own ideas while also recognizing the importance of collaborative work within a team environment. Competitive programming contestant who likes to learn new data structures and algorithms.
\begin{paracol}{2} % Begin two-column mode

%----------------------------------------------------------------------------------------
%	YOUR NAME AND CURRICULUM VITAE TITLE
%----------------------------------------------------------------------------------------

%\parbox[][0.11\textheight][c]{\linewidth}{ % Box to hold your name and CV title; change the fixed height as needed to match the colored box to the right
%	\centering % Horizontally center text
%	
%	{\sffamily\Huge Luis Miguel Casañ González} % Your name
%	
%	\medskip % Vertical whitespace
%	
%	{\Huge\textcolor{headings}{Curriculum Vitae}}
%	
%	\vfill % Push content to the top of the box
%}

%----------------------------------------------------------------------------------------
%	MAJOR RESEARCH PROJECT
%----------------------------------------------------------------------------------------

%\section{Doctoral Research}

%{\raggedright\textbf{``Observation of Einstein-Podolsky-Rosen Entanglement on Supraquantum Structures by Induction Through Nonlinear Transuranic Crystal of Extremely Long Wavelength Pulse from Mode-Locked Source Array"}\par}

%\medskip % Vertical whitespace

%My research examined the use of ELW pulses from a mode-locked source array inducted through transuranic crystals to observe entanglement on supraquantum structures. Theoretical advancements included prediction of quantum resonance phenomena including the possibility of resonance cascades. I was motivated to conduct this doctoral research due to my passion for teleportation of matter and I believe I have laid the foundation for further experimental validation and development of practical outcomes.

%\medskip % Extra vertical whitespace before the next section

%----------------------------------------------------------------------------------------
%	WORK EXPERIENCE
%----------------------------------------------------------------------------------------

\section{Work Experience}

% Each job is added with a \jobentry command. Below is an empty one to use as a template:

%\jobentry
%	{} % Duration
%	{} % FT/PT (full time or part time)
%	{} % Employer
%	{} % Job title
%	{} % Description

% All 5 parameters must be supplied but any can be empty if you don't need them

%------------------------------------------------

\jobentry
	{Current, from Apr 2021} % Duration
	{PT} % FT/PT (full time or part time)
	{} % Employer
	{\href{https://www.cent.uo.edu.cu/cbm/483-2/}{Serious Game Development}} % Job title
	{I collaborated as part of my extracurricular student work with the development team at the Medical Biophysics Center of Santiago de Cuba. Contributed to the design and implementation of a serious game application. Utilized the Unity3D graphic engine to create the entire game environment and harnessed the functionalities offered by the .NET platform to develop a seamless API, facilitating communication between different components of the system. This experience enhanced my skills in game development, software engineering, and teamwork}

\jobentry
	{from Apr 2022 - July 2022} % Duration
	{PT} % FT/PT (full time or part time)
	{}
	{Junior Web Developer at UO}
	{Collaborated closely with a professor to undertake a comprehensive web development project as part of the final course requirements. I participated in the design, structuring, and full-scale implementation, acquiring goof skills in web development technologies and methodologies. Additionally, worked collaboratively with fellow students to enhance software performance and we conducted detailed bug analysis. This experience honed my web development skills, teamwork abilities, and problem-solving acumen while delivering a successful project outcome}

\jobentry
	{from Aug, 2022 - Dec 2022} % Duration
	{PT} % FT/PT (full time or part time)
	{}
	{Desktop App Developer}
	{I worked as a freelancer for a private company developing and implementing a desktop application that would manage customer requests and automate the web marketing process.}

\jobentry
	{from Mar 2023 - July 2023} % Duration
	{PT} % FT/PT (full time or part time)
	{}
	{Machine Learning Project Developer}
	{During the culminating project of the Machine Learning course, I used deep learning techniques to implement a Convolutional Neural Network (CNN) using TensorFlow. The primary objective was to engineer a robust model capable of accurately detecting biological age from biometric data extracted from facial images}

%----------------------------------------------------------------------------------------
%	SKILLS DESCRIPTION
%----------------------------------------------------------------------------------------

\section{Skills}

\subsection{Goal Oriented}

I believe in action over long-winded discussions. I listen to everyone's viewpoints and use my judgement to immediately act based on consensus to achieve goals quickly and efficiently.

\subsection{Passionate}
I have been interested in mathematics and digital systems electronics since I was very young. My education, training and acquired research skills have cemented this interest into a passion. I very much enjoy doing mathematical research with possible practical applications.

%------------------------------------------------


%----------------------------------------------------------------------------------------

\switchcolumn % Switch to the second (right) column

%----------------------------------------------------------------------------------------
%	COLORED CONTACT DETAILS BOX
%----------------------------------------------------------------------------------------

% \parbox[top][0.11\textheight][c]{\linewidth}{ % Box to hold the colored box; change the fixed height as needed to match the box to the left
% 	\colorbox{shade}{ % Create colored box and specify background color
% 		\begin{supertabular}{@{\hspace{3pt}} p{0.05\linewidth} | p{0.775\linewidth}} % Start a table with two columns, the table will ensure everything is aligned
% 			\raisebox{-1pt}{\faHome} & Santiago de Cuba, Cuba \\ % Address
% 			\raisebox{-1pt}{\faPhone} & +53 54-725-584 \\ % Phone number
% 			\raisebox{-1pt}{\small\faEnvelope} & \href{mailto:lcasan120100@gmail.com}{lcasan120100@gmail.com} \\ % Email address
% 			%\raisebox{-1pt}{\small\faDesktop} & \href{https://www.LaTeXTemplates.com}{https://www.LaTeXTemplates.com} \\ % Website
% 			%\raisebox{-1pt}{\faGithub} & \href{https://github.com/username}{https://github.com/lcasan} \\ % GitHub profile
% 			%\raisebox{-1pt}{\faLinkedinSquare} & \href{https://www.linkedin.com/in/luis-miguel-casañ-1b4a05207}{https://www.linkedin.com/in/luis-miguel-casañ-1b4a05207} \\ % LinkedIn profile
% 			% See fontawesome.pdf in the Fonts folder for all icons you can use
% 		\end{supertabular}
% 	}
% 	%\vfill % Push content to the top of the box
% }
%----------------------------------------------------------------------------------------
%	EDUCATION
%----------------------------------------------------------------------------------------

\section{Courses Education} 

% Each qualification entry is added with a \qualificationentry command. Below is an empty one to use as a template:

%\qualificationentry
%	{} % Duration
%	{} % Degree
%	{} % Honors, achievements or distinctions (e.g. first class honors)
%	{} % Department
%	{} % Institution

% All 5 parameters must be supplied but any can be empty if you don't need them

%------------------------------------------------

\begin{supertabular}{r l} % Start a table with two columns, the table will ensure everything is aligned

	\tableentry{First Year}{ Mathematical Analysis, Algebra, Logic,}{}
	\tableentry{}{C++ Programing}{}
	\tableentry{Second Year}{Machine Architecture,}{}
	\tableentry{}{Operating System, Databases,}{}
	\tableentry{}{Probabilities, Differential Equations,}{}
	\tableentry{}{Data Structures}{}
	\tableentry{Third Year}{Computer Networks, Compilers,}{}
	\tableentry{}{Statistics, Software Engineering, }{}
	\tableentry{}{Design and Analysis of Algorithms,}{}
	\tableentry{}{Information Retrieval Systems,}{}
	\tableentry{}{Simulation, Programing Languages,}{}
	\tableentry{Fourth Year}{Artificial Intelligence, Machine Learning,}{}
	\tableentry{}{Distributed Systems, Optimization}{}

\end{supertabular}

%----------------------------------------------------------------------------------------
%	AWARDS
%----------------------------------------------------------------------------------------

\section{Awards}
\begin{supertabular}{r l} % Start a table with two columns, the table will ensure everything is aligned
	
	%------------------------------------------------
	
	\tableentry{2016}{\textbf{Silver Medal}}{}	
	\tableentry{}{\textit{National Mathematical Olympiad}}{spaceafter}
	
	%------------------------------------------------
	
	\tableentry{2017}{\textbf{Silver Medal}}{}	
	\tableentry{}{\textit{National Mathematical Olympiad}}{spaceafter}
	
	%------------------------------------------------
	
	\tableentry{2018}{\textbf{Silver Medal}}{}	
	\tableentry{}{\textit{IV Regional Mathematical Olympiad CUJAE}}{spaceafter}
\end{supertabular}

%----------------------------------------------------------------------------------------
%	COMPUTER SKILLS
%----------------------------------------------------------------------------------------

\section{Computer Skills} 

% This section is laid out using a table. A \tableentry command adds lines with the following parameters:

%\tableentry{Heading}{Content}{spaceafter}
% All 3 parameters must be supplied but any can be empty if you don't need them
% A "spaceafter" value in the third parameter will add some vertical space -- this is to be used between headings, leave it empty for no extra space

%------------------------------------------------

\begin{supertabular}{r l} % Start a table with two columns, the table will ensure everything is aligned
	\tableentry{Development Languages}{Python, C/C++, C\#}{}
	\tableentry{}{Bash, Powershell}{}
	%------------------------------------------------
	\tableentry{Development Tools}{Flask, .Net, Docker,Git,}{}
	\tableentry{}{Unity3D, TensorFlow,}{}
	\tableentry{}{Pytorch, Octave}{}
	\tableentry{Databases}{PostgreSQL, MariaDB,}{}
	\tableentry{}{SQLITE}{}
	\tableentry{Languages}{Spanish, English}{}
	%------------------------------------------------
	%\tableentry{Expert}{Perl, Unix, \LaTeX}{spaceafter}
	%------------------------------------------------
\end{supertabular}


%----------------------------------------------------------------------------------------
%	REFERENCES
%----------------------------------------------------------------------------------------

\section{Certificates}

\begin{supertabular}{r l} % Start a table with two columns, the table will ensure everything is aligned
	
	\tableentry{}{\textbf{Machine Learning}}{spaceafter}
	\tableentry{platform}{Coursera}{}
	\tableentry{skills}{Predictive Model Development, Model Evaluation, }{}
	\tableentry{}{Data Preprocessing, GNU Octave}{}
	\tableentry{}{\href{https://www.coursera.org/account/accomplishments/certificate/63UL9UVCBDRW}{\framebox[4cm][c]{Mostrar Credenciales}}}{spaceafter}

	\tableentry{}{\textbf{Python for Data Science and AI}}{spaceafter}
	\tableentry{platform}{Coursera}{}
	\tableentry{skills}{Deep Learning, Machine Learning, Data Analysis}{}
	\tableentry{}{Data Analysis}{}
	\tableentry{}{\href{https://www.credly.com/badges/b9006e81-0b04-4518-9e81-91851c22af04?source=linked_in_profile}{\framebox[4cm][c]{Mostrar Credenciales}}}{spaceafter}

\end{supertabular}

\medskip % Extra vertical whitespace before the next section

%----------------------------------------------------------------------------------------
%	PUBLICATIONS
%----------------------------------------------------------------------------------------

%\section{Publications}
%
%%------------------------------------------------
%
%\textbf{Freeman, G. R.} (1996). Chemistry of Multiply Charged Negative Molecular Ions and Clusters in the Gas Phase:  Terrestrial and in Intense Galactic Magnetic Fields. \textit{The Journal of Physical Chemistry}, \textit{100}(11), 4331-4338.
%
%\medskip % Vertical whitespace
%
%Jacobsen, F. M., Gee, N., \textbf{Freeman, G. R.} (1986). Electron mobility in liquid krypton as function of density, temperature, and electric field strength. \textit{Physical Review A}, \textit{34}(3): 2329-2335.
%
%\medskip % Vertical whitespace

%------------------------------------------------

% As an alternative to a long-form publication list, you can create a shorter summary using only DOI values and years.

% Example \doipublication{} command to add another publication:

%\doipublication{Year}{DOI}{firstauthor}{spaceafter}

% All four parameters are required (can be empty though)
% A value of "firstauthor" in the third parameter will output the DOI in bold
% A "spaceafter" value in the fourth parameter will add some vertical space -- this is to be used between years

%------------------------------------------------

%\subsection{Publications by DOI}
%
%\begin{supertabular}{r l} % Start a table with two columns, the table will ensure everything is aligned
%	
%	%------------------------------------------------
%	
%	\doipublication{1996}{10.1021/jp951483+}{firstauthor}{spaceafter}
%	
%	%------------------------------------------------
%	
%	\doipublication{1990}{10.1139/p90-097}{firstauthor}{spaceafter}
%	
%	%------------------------------------------------
%	
%	\doipublication{1986}{10.1139/v86-297}{}{}
%	\doipublication{}{10.1103/PhysRevA.34.2329}{}{spaceafter}
%	
%	%------------------------------------------------
%	
%	& \textit{First author publications in} \textbf{bold}\\
%	
%	%------------------------------------------------
%	
%\end{supertabular}

%\medskip % Extra whitespace before the next section

%----------------------------------------------------------------------------------------

\end{paracol} % End two-column mode

%----------------------------------------------------------------------------------------

\end{document}
